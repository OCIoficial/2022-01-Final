\documentclass{oci}
\usepackage[utf8]{inputenc}
\usepackage{lipsum}

\title{Ocilandia}

\begin{document}
\begin{problemDescription}
  Pedrito acaba de cumplir su sueño de visitar Ocilandia, el parque de diversiones más famoso y
  popular del mundo.
  Pedrito estará en el parque solo por un día y quiere usar su tiempo de la forma más eficiente posible.

  Ocilandia no es un parque de diversiones común y corriente.
  El parque tiene $N$ atracciones y cada una entrega una experiencia individual adaptada a
  las necesidades de cada visitante.
  Por esta razón, el tiempo que tarda visitar una atracción varía de persona a persona.

  Cada una de las atracciones tiene una fila de personas esperando poder visitarla.
  La $i$-ésima atracción tiene una fila de largo $M_i$.
  Adicionalmente, para cada atracción, sabemos el tiempo que tardará cada persona en la fila en
  visitar la atracción.
  El \emph{tiempo de espera} de una atracción es igual a la suma de los tiempos que tardarán
  cada una de las personas en la fila en visitar la atracción.

  Pedrito quiere usar bien su tiempo y está interesado en saber el menor tiempo de espera entre
  todas las atracciones.
  ?`Podrías ayudarlo?
\end{problemDescription}

\begin{inputDescription}
  La primera línea de la entrada contiene un entero $N$ ($0 < N \leq 1000$) correspondiente a la
  cantidad de atracciones.

  La siguiente línea contiene $N$ enteros indicando el largo de la fila para cada atracción.
  El entero $i$-ésimo corresponde al largo $M_i$ ($0 < M_i \leq 1000$) de la fila en la $i$-ésima atracción.

  Posteriormente, vienen $N$ líneas describiendo las filas para cada una de las atracciones.
  La $i$-ésima línea contiene $M_i$ enteros cada uno indicando el tiempo que tardará cada persona
  en visitar la atracción.
\end{inputDescription}

\begin{outputDescription}
  La salida debe contener una línea con un entero indicando el tiempo mínimo de espera entre todas
  las atracciones.
\end{outputDescription}

\begin{scoreDescription}
  Este problema no contiene subtareas.
  Se entregará puntaje proporcional a la cantidad de casos de prueba correctos,
  siendo 100 el puntaje máximo.
\end{scoreDescription}

\begin{sampleDescription}
\sampleIO{sample-1}
% \sampleIO{sample-2}
\end{sampleDescription}

\end{document}
