\documentclass{oci}
\usepackage[utf8]{inputenc}
\usepackage{tikz}

\newcommand{\NS}[1]{\texttt{#1NS}}
\renewcommand{\OE}[1]{\texttt{#1OE}}

\definecolor{blue}{RGB}{30, 136, 229}
\definecolor{yellow}{RGB}{255, 193, 7}
\definecolor{red}{RGB}{216, 27, 96}

\title{Buses}

\begin{document}
\begin{problemDescription}
  Décadas de malas políticas de desarrollo han condenado a los habitantes de Grilland a depender
  de automóviles particulares para poder desplazarse por la ciudad.
  Dado el indiscutible impacto negativo que esto tiene en el medio ambiente, los habitantes de
  Grilland han decidido poner un alto a la dependencia al automóvil.
  Los expertos saben que la mejor forma de lograr este objetivo es hacer que la ciudad sea más
  \emph{caminable}.
  Como primer paso, las autoridades están interesadas en saber qué puntos de la ciudad son
  actualmente los más caminables.
  Un punto es más caminable mientras más \emph{puntos de interés} sea posible alcanzar caminando
  desde él.

  La geometría de Grilland es muy especial.
  Sus calles forman un grilla perfecta con $M$ calles en dirección horizontal (oeste-este),
  y $N$ calles en dirección vertical (norte-sur).
  Las calles en dirección horizontal son numeradas de norte a sur entre 1 y $M$.
  Las calles en dirección norte-sur son numeradas de oeste a este entre 1 y $N$.
  Dada una calle $i$ en dirección horizontal y una calle $j$ en dirección vertical identificamos
  con el par $(i, j)$ a la intersección entre ambas calles.
  Denominamos \emph{cuadra} al segmento de una calle contenido entre dos intersecciones consecutivas.
  La siguiente imagen muestra un ejemplo de la geometría de Grilland para $M=5$ y $N=7$, donde se
  ha marcado con {\bf\color{blue}azul} la cuadra entre las intersecciones (2, 3) y (2, 4), y con
  {\bf\color{red}rojo} la cuadra entre las intersecciones (3, 6) y (4, 6).


  \begin{center}
  \scalebox{0.8}{
    \begin{tikzpicture}
      \def\M{4}
      \def\N{6}
      \draw[gray] (0, 0) grid (\N,\M);
      \foreach \x [count=\i from 1] in {0, ..., \N} {
        \node at (\x, \M + 0.4) {\i};
      }
      \foreach \y [count=\i from 1] in {0, ..., \M} {
        \node at (-0.4, \M - \y) {\i};
      }
      \draw[line width=1mm, blue] (2, 3) to (3, 3);
      \draw[line width=1mm, red]  (5, 1) to (5, 2);
      \node at (\N/2, \M + 1) {\bf N};
      \node at (\N/2, -1)     {\bf S};
      \node at (-1, \M/2)     {\bf O};
      \node at (\N + 1, \M/2) {\bf E};
      % \foreach \x [count=\i from 1] in {0, ..., \N} {
      %   \foreach \y [count=\i from 1] in {0, ..., \M} {
      %     \node[circle, draw, fill=white, scale=0.5] at (\x, \y) {};
      %   }
      % }
    \end{tikzpicture}
  }
  \end{center}

  % El nombre de cada calle está compuesto por su número y el sufijo \OE{} para las calles en dirección
  % oeste-este y el sufijo \NS{} para las calles en dirección norte-sur.
  % Por ejemplo, el nombre de la tercera calle en dirección oeste-este es \OE{3} y la quinta
  % calle en la dirección norte-sur es \NS{5}.

  \begin{center}
  \scalebox{0.8}{
    \begin{tikzpicture}
      \def\M{6}
      \def\N{11}
      \draw (0, 0) grid (\N,\M);
      \foreach \x [count=\i from 1] in {0, ..., \N} {
        \node at (\x, \M + 0.5) {\i};
      }
      \foreach \y [count=\i from 1] in {0, ..., \M} {
        \node at (-0.5, \M - \y) {\i};
      }
      \node at (\N/2, \M + 1) {\bf N};
      \node at (\N/2, -1)     {\bf S};
      \node at (-1, \M/2)     {\bf O};
      \node at (\N + 1, \M/2) {\bf E};
      % \foreach \x [count=\i from 1] in {0, ..., \N} {
      %   \foreach \y [count=\i from 1] in {0, ..., \M} {
      %     \node[circle, draw, fill=white, scale=0.5] at (\x, \y) {};
      %   }
      % }
    \end{tikzpicture}
  }
  \end{center}


\end{problemDescription}

\begin{inputDescription}
  La primera línea de la entrada contiene cuatro enteros $M$ y $N$ correspondientes respectivamente
  a la cantidad de calles en dirección horizontal y la cantidad de calles en dirección vertical.
  La segunda línea contiene $M$ enteros describiendo si las calles en dirección horizontal
  contienen o no una vereda.
  El entero $i$-ésimo será un 1 si la calle $i$ tiene vereda o un 0 en caso contrario.
  Similarmente, la tercera línea contiene $N$ enteros describiendo si las calles en dirección
  vertical contienen veredas.

  A continuación sigue una línea con un entero $E$ correspondiente a la cantidad de puntos de interés.
  Las siguientes $E$ líneas describen cada una la cuadra en que se encuentra cada punto de interés.
  Cada línea contiene tres enteros $a$, $b$, $c$.
  El par $(a, b)$ representa la intersección de inicio de la cuadra.
  Si el entero $c$ es 0, la cuadra está en dirección horizontal, es decir la cuadra está contenida entre las
  intersecciones $(a, b)$ y $(a, b + 1)$.
  Si $c$ es 1, la cuadra está orientada en dirección vertical, es decir, la cuadra está contendida entre
  las intersecciones $(a, b)$ y $(a + 1, b)$.
  Se garantiza que la cuadra siempre estará contenida en la grilla.

  Finalmente, la última línea contiene 3 enteros $X$, $Y$ y $F$.
  El par $(X, Y)$ describe la intersección inicial.
  El entero $F$ describe la cantidad máxima de cuadras.
\end{inputDescription}

\begin{outputDescription}
  La salida debe contener un único entero indicando la cantidad de puntos de interés que pueden
  ser alcanzados recorriendo $F$ o menos cuadras.
\end{outputDescription}

\begin{scoreDescription}
  \subtask{20}
  Descripción Subtarea1
  \subtask{30}
  Descripción Subtarea2
  \subtask{50}
  Descripción Subtarea3
\end{scoreDescription}

\begin{sampleDescription}
\sampleIO{sample-1}
\sampleIO{sample-2}
\end{sampleDescription}

\end{document}
