\documentclass[12pt]{oci}
\usepackage{graphicx}
\usepackage{enumitem}
\usepackage[utf8]{inputenc}

\thispagestyle{empty}
\begin{document}

  \begin{center}
  \includegraphics[height=70pt]{logo.eps}

  \vskip 70pt
  \Large{\bf Olimpiada Chilena de Inform\'atica\\ \the\year}
  \vskip 10pt
  \large{\phase}
  \vskip 10pt
  \normalsize{\it 22 de Enero, \the\year}

  \vskip 85pt

  % \emph{Las siguientes personas participaron en la elaboración de este conjunto
  %   de problemas:}
  \vskip 10pt
\end{center} % \end{itemize}

\cleardoublepage


\subsection*{Información General}

Esta página muestra información general que se aplica a todos los problemas.

\subsection*{Envío de una solución}

\begin{enumerate}
% \itemsep 0em
\item Los participantes deben enviar {\bf un solo archivo} con el código fuente de su solución.
\item El nombre del archivo debe tener la extensión \verb+.cpp+ o
  \verb+.java+ dependiendo de si la solución está escrita en
  \verb|C++| o \verb|Java| respectivamente.
Para enviar una solución en Java hay que seguir algunos pasos adicionales. Ver detalles más abajo.
\end{enumerate}

\subsection*{Casos de prueba, subtareas y puntaje}
\begin{enumerate}
  % \itemsep 0em
\item La solución enviada por los participantes será ejecutada varias veces con
  distintos casos de prueba.
\item A menos que se indique lo contrario, cada problema define diferentes
  subtareas que lo restringen. Se asignará puntaje de acuerdo a la
  cantidad de subtareas que se logre solucionar de manera correcta.
\item A menos que se indique lo contrario, para obtener el puntaje en una
  subtarea se debe tener correctos todos los casos de prueba incluídos en ella.
\item Una solución puede resolver al mismo tiempo más de una subtarea.
\item La solución es ejecutada con cada caso de prueba de manera independiente y
  por tanto puede fallar en algunas subtareas sin influir en la ejecución de
  otras.
\end{enumerate}

\subsection*{Entrada}
\begin{enumerate}
% \itemsep 0em
\item Toda lectura debe ser hecha desde la {\bf entrada estándar} usando, por
  ejemplo, las funciones \verb+scanf+ o \verb+std::cin+ en C++ o la clase
  \verb+BufferedReader+ en Java.
\item La entrada corresponde a un solo caso de prueba, el cual está descrito en
  varias líneas dependiendo del problema.
\item {\bf Se garantiza que la entrada sigue el formato descrito} en el
  enunciado de cada problema.
\end{enumerate}

\newpage
\subsection*{Salida}
\begin{enumerate}
% \itemsep 0em
\item Toda escritura debe ser hecha hacia la {\bf salida estándar} usando, por
  ejemplo, las funciones \verb+printf+, \verb+std::cout+ en C++  o
  \verb+System.out.println+ en Java.
\item El formato de salida es explicado en el enunciado de cada problema.
\item {\bf La salida del programa debe cumplir estrictamente con el formato indicado},
  considerando los espacios, las mayúsculas y minúsculas.
\item Toda línea, incluyendo la última, debe terminar con un salto de línea.
\end{enumerate}

% \newpage

\subsection*{Envío de una solución en Java}

\begin{enumerate}
  \item Cada problema tiene un \emph{nombre clave} que será especificado en el
    enunciado.
    Este nombre clave será también utilizado en el sistema de evaluación
    para identificar al problema.
  \item Para enviar correctamente una solución en Java, el archivo debe contener
    una clase llamada igual que el nombre clave del problema.
    Esta clase debe contener también el método \verb+main+.
    Por ejemplo, si el nombre clave es \texttt{marraqueta}, el archivo con la
    solución debe llamarse \texttt{marraqueta.java} y tener la siguiente
    estructura:

\begin{verbatim}
public class marraqueta {
  public static void main (String[] args) {
    // tu solución va aquí
  }
}
\end{verbatim}
  \item Si el archivo no contiene la clase con el nombre correcto, el sistema de
    evaluación reportará un error de compilación.

  \item La clase no debe estar contenida dentro de un \emph{package}.
      Hay que tener cuidado pues algunos entornos de desarrollo como Eclipse
    incluyen las clases en un \emph{package} por defecto.
  \item Si la clase está contenida dentro de un package, el sistema reportará un
    error de compilación.
\end{enumerate}


\end{document}
